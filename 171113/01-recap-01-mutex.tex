\section{Напоминание}
\subsection{Потоки, гонки, мьютексы}

\begin{frame}[t,fragile]{Жизненный цикл потоков}
	\svgimg{01-recap-01-mutex-01}
\end{frame}

\begin{frame}[t,fragile]{Потоков бывает много}
	\svgimg{01-recap-01-mutex-02}
\end{frame}

\begin{frame}[t]{Напоминание про потоки}
	\begin{itemize}
		\item Потоки выполняют код независимо и параллельно
		\item Так удобно писать код, который работает над несколькими вещами одновременно
		\item Если несколько ядер процессора "--- ещё и получается быстрее
		\item Внутри одного \textit{процесса} работает несколько \textit{потоков}
		\item У всех потоков внутри процесса общая память
		\item Но лучше к общей памяти не обращаться (см. далее "--- \textit{гонки})
	\end{itemize}
\end{frame}

\begin{frame}[t,fragile]{Гонка данных (повезло)}
	\svgimg{01-recap-01-mutex-03}
\end{frame}

\begin{frame}[t,fragile]{Гонка данных (не повезло)}
	\svgimg{01-recap-01-mutex-04}
\end{frame}

\begin{frame}[t,fragile]{Гонка ресурсов (повезло)}
	\svgimg{01-recap-01-mutex-05}
\end{frame}

\begin{frame}[t,fragile]{Гонка ресурсов (не повезло)}
	\svgimg{01-recap-01-mutex-05b}
\end{frame}

\begin{frame}[t,fragile]{Гонка ресурсов (как правильно)}
	\svgimg{01-recap-01-mutex-06}
\end{frame}

\begin{frame}[t]{Напоминание про гонки}
	\begin{itemize}
		\item Пока не знаем ничего атомарного, кроме \textit{захвата} или \textit{освобождения} мьютекcов
		\item Если хотим сделать операцию атомарной "--- \textit{защищаем} мьютексом
		\item Операции между разными потоками могут как угодно перемешиваться
		\item Между двумя атомарными операциями могут вклиниться другие, если не защитить
		\item Захват и освобождение "--- медленные операции
	\end{itemize}
\end{frame}
