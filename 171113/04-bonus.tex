\section{Бонус}

\begin{frame}
	\tableofcontents[currentsection,currentsubsection]
\end{frame}

\begin{frame}{Как отлаживать}
	\begin{center}
		\includegraphics[scale=0.2]{apple-a-day.jpg}
		\pause

		An apple a day keeps the doctor away.

		Лучше предотвращать, чем отлаживать.
	\end{center}
\end{frame}

\begin{frame}{Причина}
	\begin{center}
		\includegraphics[scale=0.2]{race-or-bug.jpg}
	\end{center}
	\begin{itemize}
		\item Многопоточные баги обычно тесно связаны с порядком выполнения операций.
		\item Операции выполняются в разном порядке каждый запуск, под отладчиком, в разном коде.
		\item Очень сложно ловить баг <<за руку>>.
		\item Корректная работа на куче тестов не означает отсутствие багов.
	\end{itemize}
\end{frame}

\begin{frame}{Как предотвращать}
	\begin{itemize}
		\item Явно расставляйте инварианты в комментариях: что чем защищено, в каком порядке захватывать мьютексы.
		\item Нарисуйте на бумажке все возможные состояния системы и проверьте, что инварианты выполняются.
		\item Минимизируйте количество мьютексов, если нет проблем со скоростью работы.
		\item Не используйте для синхронизации ничего, кроме мьютексов (в частности, явных \t{sleep} в программе быть не должно).
	\end{itemize}
\end{frame}

\begin{frame}{Как тестировать}
	\begin{itemize}
		\item Запускайте на больших тестах, в которых потоки работают медленно и часто происходит переключение.
		\item
			Если вы под 64-битным Linux "--- используйте thread sanitizer (добавьте ключи \t{-g -fsanitize=thread -O2 -fPIC -pie}).
			Он хорош в нахождении некоторых гонок данных, \textit{происходящих во время выполнения}.
		\item
			Для аналогичных целей можно использовать Valgrind (задавайте вопросы!).
		\item
			На Windows можно поставить виртуальную машину.
	\end{itemize}
\end{frame}

\begin{frame}[fragile]{Пример вывода Thread Sanitizer-1}
	На \href{https://github.com/yeputons/fall-2017-paradigms/raw/master/171023/sources/08-two-threads.cpp}{примере с двумя счётчиками} находит гонку сразу,
	во время первых выводов на экран:
\begin{verbatim}
==================
WARNING: ThreadSanitizer: data race (pid=13170)
  Read of size 4 at 0x7f31e00e12cc by thread T2:
    #0 worker .../08-two-threads.c:10 (...)
    #1 <null> <null>:0 (libtsan.so.0+0x000000032d69)

  Previous write of size 4 at 0x7f31e00e12cc by thread T1:
    #0 worker .../08-two-threads.c:12 (...)
    #1 <null> <null>:0 (libtsan.so.0+0x000000032d69)

  Location is global 'data' of size 4 at ... (...)
\end{verbatim}
\end{frame}

\begin{frame}[fragile]{Пример вывода Thread Sanitizer-2}
	Также указывает, где были созданы соответствующие потоки:
\begin{verbatim}
  Thread T2 (tid=13173, running) created by main thread at:
    #0 pthread_create <null>:0 (libtsan.so.0+0x000000047f23)
    #1 main .../08-two-threads.c:20 (...)

  Thread T1 (tid=13172, finished) created by main thread at:
    #0 pthread_create <null>:0 (libtsan.so.0+0x000000047f23)
    #1 main .../08-two-threads.c:19 (...)

SUMMARY: ThreadSanitizer: data race .../08-two-threads.c:10
\end{verbatim}
\end{frame}
