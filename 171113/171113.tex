\documentclass[utf8,xcolor=table]{beamer}

\usepackage[T2A]{fontenc}
\usepackage[utf8]{inputenc}
\usepackage[english,russian]{babel}
\usepackage{minted}
\usepackage{ulem}
\usepackage{cmap}
\usepackage{multirow}

\hypersetup{colorlinks,linkcolor=blue,urlcolor=blue}

\mode<presentation>{
	\usetheme{CambridgeUS}
}

\renewcommand{\t}[1]{\ifmmode{\mathtt{#1}}\else{\texttt{#1}}\fi}

\title{Многопоточность-2}
\author{Егор Суворов}
\institute[СПб АУ]{Курс <<Парадигмы и языки программирования>>, подгруппа 3}
\date[13.11.2017]{Понедельник, 13 ноября 2017 года}

\setlength{\arrayrulewidth}{1pt}

\begin{document}

\begin{frame}
\titlepage
\end{frame}

\begin{frame}{План занятия}
	\tableofcontents
\end{frame}

\section{Напоминание}
\subsection{Потоки, гонки, мьютексы}

\input{01-recap-02-dont-try}
\input{02-condvar-01-spinlock}
\input{02-condvar-02-condvar}
\input{03-bonus}
\input{04-hometask}

\end{document}
