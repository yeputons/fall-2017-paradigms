\documentclass[utf8,xcolor=table]{beamer}

\usepackage[T2A]{fontenc}
\usepackage[utf8]{inputenc}
\usepackage[english,russian]{babel}
\usepackage{minted}
\usepackage{ulem}
\usepackage{cmap}
\usepackage{multirow}

\hypersetup{colorlinks,linkcolor=blue,urlcolor=blue}

\mode<presentation>{
	\usetheme{CambridgeUS}
}

\renewcommand{\t}[1]{\ifmmode{\mathtt{#1}}\else{\texttt{#1}}\fi}
\newcommand{\svgimg}[1]{
  \begin{center}
	\includegraphics[width=\textwidth,height=0.8\textheight,keepaspectratio]{#1.pdf}
  \end{center}
}


\title{Многопоточность-2}
\author{Егор Суворов}
\institute[СПб АУ]{Курс <<Парадигмы и языки программирования>>, подгруппа 3}
\date[13.11.2017]{Понедельник, 13 ноября 2017 года}

\setlength{\arrayrulewidth}{1pt}

\begin{document}

\begin{frame}
\titlepage
\end{frame}

\begin{frame}{План занятия}
	\tableofcontents
\end{frame}

\section{Напоминание}
\subsection{Потоки, гонки, мьютексы}

\input{01-recap-02-dont-try}
\input{02-condvar-01-spinlock}
\input{02-condvar-02-condvar}
\input{03-hometask}
\section{Бонус}

\begin{frame}
	\tableofcontents[currentsection,currentsubsection]
\end{frame}

\begin{frame}{Стандартные паттерны-1}
	\begin{itemize}
		\item
			\textbf{Пул потоков} (thread pool):
			\begin{itemize}
				\item Создание потоков на короткие задачи "--- это очень неэффективно.
				\item Поддерживается пул из некоторого числа потоков (примерно по числу ядер).
				\item Задачу можно отправить в пул, она выполнится в одном из потоков, когда тот освободится.
				\item Ограничивает число одновременно выполняющихся задач.
			\end{itemize}
		\item
			\textbf{Блокировка чтения-записи} (readers-writer lock)
			\begin{itemize}
				\item Как мьютекс, но позволяет потоку указать режим доступа: <<чтение>> или <<запись>>.
				\item Читателей может быть сколько угодно.
				\item Если кто-то пишет, то другие потоки ждут.
				\item Ускоряет доступ к редко меняющимся данным.
			\end{itemize}
	\end{itemize}
\end{frame}

\begin{frame}{Стандартные паттерны-2}
	\begin{itemize}
		\item
			\textbf{Акторы} (actors)
			\begin{itemize}
				\item Небольшие потоки, не имеющие общих ресурсов.
				\item Для обмена информацией посылают друг другу неизменяемые \textit{сообщения}.
				\item Вся синхронизация сконцентрирована в системе обмена сообщениями.
			\end{itemize}
		\item
			\textbf{Неблокирующие} (lock-free) структуры данных
			\begin{itemize}
				\item Работают поверх атомарных операций и моделей памяти.
				\item Не требуют блокировок или мьютексов.
				\item Работают быстрее, так как многие операции атомарны в железе и не требуют вмешательства ОС.
			\end{itemize}
	\end{itemize}
\end{frame}

\begin{frame}{Как отлаживать}
	\begin{center}
		\includegraphics[scale=0.2]{apple-a-day.jpg}
		\pause

		An apple a day keeps the doctor away.

		Лучше предотвращать, чем отлаживать.
	\end{center}
\end{frame}

\begin{frame}{Причина}
	\begin{center}
		\includegraphics[scale=0.2]{race-or-bug.jpg}
	\end{center}
	\begin{itemize}
		\item Многопоточные баги обычно тесно связаны с порядком выполнения операций.
		\item Операции выполняются в разном порядке каждый запуск, под отладчиком, в разном коде.
		\item Очень сложно ловить баг <<за руку>>.
		\item Корректная работа на куче тестов не означает отсутствие багов.
	\end{itemize}
\end{frame}

\begin{frame}{Как предотвращать}
	\begin{itemize}
		\item Явно расставляйте инварианты в комментариях: что чем защищено, в каком порядке захватывать мьютексы.
		\item Нарисуйте на бумажке все возможные состояния системы и проверьте, что инварианты выполняются.
		\item Минимизируйте количество мьютексов, если нет проблем со скоростью работы.
		\item Не используйте для синхронизации ничего, кроме мьютексов (в частности, явных \t{sleep} в программе быть не должно).
	\end{itemize}
\end{frame}

\begin{frame}{Как тестировать}
	\begin{itemize}
		\item Запускайте на больших тестах, в которых потоки работают медленно и часто происходит переключение.
		\item
			Если вы под 64-битным Linux "--- используйте thread sanitizer (добавьте ключи \t{-g -fsanitize=thread -O2 -fPIC -pie}).
			Он хорош в нахождении некоторых гонок данных, \textit{происходящих во время выполнения}.
		\item
			Для аналогичных целей можно использовать Valgrind.
		\item
			На Windows можно поставить виртуальную машину.
	\end{itemize}
\end{frame}

\begin{frame}[fragile]{Пример вывода Thread Sanitizer-1}
	На \href{https://github.com/yeputons/fall-2016-paradigms/raw/master/161019/sources/08-two-threads.c}{примере с двумя счётчиками} находит гонку сразу,
	во время первых выводов на экран:
\begin{verbatim}
==================
WARNING: ThreadSanitizer: data race (pid=13170)
  Read of size 4 at 0x7f31e00e12cc by thread T2:
    #0 worker .../08-two-threads.c:10 (...)
    #1 <null> <null>:0 (libtsan.so.0+0x000000032d69)

  Previous write of size 4 at 0x7f31e00e12cc by thread T1:
    #0 worker .../08-two-threads.c:12 (...)
    #1 <null> <null>:0 (libtsan.so.0+0x000000032d69)

  Location is global 'data' of size 4 at ... (...)
\end{verbatim}
\end{frame}

\begin{frame}[fragile]{Пример вывода Thread Sanitizer-2}
	Также указывает, где были созданы соответствующие потоки:
\begin{verbatim}
  Thread T2 (tid=13173, running) created by main thread at:
    #0 pthread_create <null>:0 (libtsan.so.0+0x000000047f23)
    #1 main .../08-two-threads.c:20 (...)

  Thread T1 (tid=13172, finished) created by main thread at:
    #0 pthread_create <null>:0 (libtsan.so.0+0x000000047f23)
    #1 main .../08-two-threads.c:19 (...)

SUMMARY: ThreadSanitizer: data race .../08-two-threads.c:10
\end{verbatim}
\end{frame}


\end{document}
