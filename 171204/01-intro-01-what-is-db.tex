\section{Основы основ}
\subsection{Что такое СУБД}

\begin{frame}
	\tableofcontents[currentsection,currentsubsection]
\end{frame}

\begin{frame}{Постановка задачи}
	\begin{itemize}
		\item Пусть мы пишем приложения для учёта товаров в магазине.
		\item Надо знать:
			\begin{enumerate}
				\item Какой товар есть на складе и витринах.
				\item Где он лежит.
				\item По какой цене товар закуплен (могут быть разные партии).
				\item По какой цене товар сейчас продаётся.
				\item Какие покупки были сделаны (что куплено вместе, на какую сумму, в какое время).
			\end{enumerate}
		\item Возможные события:
			\begin{enumerate}
				\item Приехала поставка со склада.
				\item Касса пробила чек "--- совершена покупка.
			\end{enumerate}
		\item Надо, чтобы приложение сохраняло состояние между перезапусками.
		\item Вопрос: как это сделать?
	\end{itemize}
\end{frame}

\begin{frame}{Усложнения}
	\begin{itemize}
		\item Запуск должен быть быстрый
		\item Данные могут не помещаться в память
		\item Может случайно отключаться электричество
		\item Десять касс и два компьютера в разных концах склада.
		\item Хочется получать обновления <<как только так сразу>>
		\item Иногда может теряться связь между кассами и складом
		\item Может меняться формат хранения (например, добавили бонусы за товар)
	\end{itemize}
\end{frame}

\begin{frame}[t]{СУБД}
	\begin{itemize}
		\item \textit{Система управления базами данных} (СУБД) "--- это сервис, который умеет хранить данные \textit{произвольной структуры}
			(в определённых рамках, конечно, не совсем бессистемные).
		\item \textit{База данных} "--- это описание данных \textbf{и их структуры}, которые хранятся в СУБД.
		\item СУБД обычно делят на два вида в зависимости от того, как они структурируют данные: реляционные (relational) и нереляционные (non-relational или NoSQL).
		\item Реляционные "--- это классика (существуют с 80-х годов), их и будем изучать.
		\item Нереляционные примерно того же возраста, но вошли в тренд только в последние лет десять.
		\item Примеры реляционных: MySQL, MariaDB, Oracle, MS SQL, Sqlite.
		\item Примеры нереляционных: MongoDB, Redis, Memcached, Cassandra.
	\end{itemize}
\end{frame}

\begin{frame}{Реляционные СУБД на практике}
	\begin{itemize}
		\item СУБД хранит одну или несколько независимых БД (баз данных).
		\item Каждая БД "--- это набор \textit{таблиц} (<<отношений>>), которые содержат данные.
		\item Таблица имеет фиксированный набор столбцов с названиями и типами.
		\item Фиксированный в каждый момент времени; вообще столбцы можно добавлять, менять, удалять, хоть это и сложные для СУБД операции.
		\item В таблице лежит неупорядоченный набор строк с данными.
		\item На столбцы (или их группы) могут накладываться дополнительные ограничения (например, <<все значения в столбце различны>>).
		\item Обычно запросы к реляционным СУБД формулируются на декларативном языке SQL
			(Structured Query Language).
	\end{itemize}
\end{frame}
