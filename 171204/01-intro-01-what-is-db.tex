\section{Основы основ}
\subsection{Что такое СУБД}

\begin{frame}
	\tableofcontents[currentsection,currentsubsection]
\end{frame}

\begin{frame}{Постановка задачи}
	\begin{itemize}
		\item Пусть мы пишем приложения для учёта товаров в магазине.
		\item Надо знать:
			\begin{enumerate}
				\item Какой товар есть на складе и витринах.
				\item Где он лежит.
				\item По какой цене товар закуплен (могут быть разные партии).
				\item По какой цене товар сейчас продаётся.
				\item Какие покупки были сделаны (что куплено вместе, на какую сумму, в какое время).
			\end{enumerate}
		\item Возможные события:
			\begin{enumerate}
				\item Приехала поставка со склада.
				\item Касса пробила чек "--- совершена покупка.
			\end{enumerate}
		\item Надо, чтобы приложение сохраняло состояние между перезапусками.
		\item Вопрос: как это сделать?
	\end{itemize}
\end{frame}

\begin{frame}{Усложнения}
	\begin{itemize}
		\item Запуск должен быть быстрый
		\item Данные могут не помещаться в память
		\item Может случайно отключаться электричество
		\item Десять касс и два компьютера в разных концах склада.
		\item Хочется получать обновления <<как только так сразу>>
		\item Иногда может теряться связь между кассами и складом
		\item Может меняться формат хранения (например, добавили бонусы за товар)
	\end{itemize}
\end{frame}

\begin{frame}[t]{СУБД}
	\begin{itemize}
		\item \textit{Система управления базами данных} (СУБД) "--- это сервис, который умеет хранить данные \textit{произвольной структуры}
			(в определённых рамках, конечно, не совсем бессистемные).
		\item \textit{База данных} "--- это описание данных \textbf{и их структуры}, которые хранятся в СУБД.
		\item СУБД обычно делят на два вида в зависимости от того, как они структурируют данные: реляционные (relational) и нереляционные (non-relational или NoSQL).
		\item Реляционные "--- это классика (существуют с 80-х годов), их и будем изучать.
		\item Нереляционные примерно того же возраста, но вошли в тренд только в последние лет десять.
		\item Примеры реляционных: MySQL, MariaDB, Oracle, MS SQL, Sqlite.
		\item Примеры нереляционных: MongoDB, Redis, Memcached, Cassandra.
	\end{itemize}
\end{frame}

\begin{frame}[t]{Где и зачем}
	СУБД используются практически везде:
	\begin{itemize}
		\only<1>{
		\item
			Если данных или клиентов (которые запрашивают/меняют данные) будет очень много,
			то нам не надо изобретать велосипед и писать своё масштабируемое хранилище:
			\begin{enumerate}
				\item Обычно одна СУБД обслуживает сразу несколько приложений.
				\item Можно создавать разных пользователей с разными правами.
				\item Можно прозрачно для приложений делать бэкапы или хранить данные на десяти серверах.
			\end{enumerate}
		}
		\only<2>{
		\item
			Если мы просто пишем приложение с какой-то нетривиальной схемой:
			\begin{enumerate}
				\item Очень чётко отделяются данные от их обработки.
				\item SQL все знают (в отличие от логики программы), легко делать запросы к БД, зная только схему, и не зная ничего про приложение.
				\item SQL мощнее и читается лучше циклов for и list comprehension, которые ещё и не во всех языках есть.
				\item Не надо думать про хранение данных.
			\end{enumerate}
		}
		\only<3>{
		\item
			Если мы data scientist и/или хотим активно проверять гипотезы и много/просто работать с данными:
			\begin{enumerate}
				\item Удобно, когда все данные лежат в БД с известным интерфейсом (SQL).
				\item Не надо писать никакой код и ни с чем интегрироваться, чтобы выполнить запрос.
				\item Не получится набагать в коде в обработке крайних случаев\footnote{Даже в SQL можно посадить сложный баг}
			\end{enumerate}
		}
	\end{itemize}
\end{frame}

\begin{frame}{В чём минусы}
	\begin{itemize}
		\item
			Мы отдаём контроль за скоростью выполнения и потреблением памяти в руки СУБД
			(как и при любой абстракции).
			Это обычно приемлимый компромисс.
		\item
			Приложение сложнее запустить: нужно настроить СУБД, что обычно занимает несколько шагов.
			В нестадартных ситуациях "--- больше.
		\item
			Иногда приложение требует слишком хитрую настройку СУБД (например, для корректной работы
			с не-латиницей и датами).
		\item
			Многие инструменты заточены под промышленные решения и имеют слишком много рычажков и кнопок
			для простых целей.
	\end{itemize}
\end{frame}

\begin{frame}{Реляционные СУБД на практике}
	\begin{itemize}
		\item СУБД хранит одну или несколько независимых БД (баз данных).
		\item Каждая БД "--- это набор \textit{таблиц} (<<отношений>>), которые содержат данные.
		\item Таблица имеет фиксированный набор столбцов с названиями и типами.
		\item Фиксированный в каждый момент времени; вообще столбцы можно добавлять, менять, удалять, хоть это и сложные для СУБД операции.
		\item В таблице лежит неупорядоченный набор строк с данными.
		\item На столбцы (или их группы) могут накладываться дополнительные ограничения (например, <<все значения в столбце различны>>).
		\item Обычно запросы к реляционным СУБД формулируются на декларативном языке SQL
			(Structured Query Language).
	\end{itemize}
\end{frame}
