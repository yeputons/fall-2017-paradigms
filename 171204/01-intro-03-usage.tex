\subsection{Использование СУБД}

\begin{frame}
	\tableofcontents[currentsection,currentsubsection]
\end{frame}

\begin{frame}{Анонс домашнего задания}
	\begin{itemize}
		\item Вам будет выдан файл с SQL-запросами, которые создают таблицы со странами (структуру разберём) и заполняют их данными.
		\item Вам нужно написать несколько SQL-запросов \t{SELECT}, которые что-то вычисляют.
		\item Тестировать можно на созданных тестовых данных.
		\item Как именно тестировать "--- сейчас покажу.
	\end{itemize}
\end{frame}

\begin{frame}{Консольная утилита}
	\begin{itemize}
		\item Называется sqlite3. Это просто программа, которая умеет выполнять SQL-запросы на БД sqlite.
		\item По умолчанию создаёт пустую БД в памяти.
		\item Можно попросить открыть существующую БД в файле (или создать новый файл).
		\item При помощи перенаправления может выполнять SQL из файла.
		\item SQL-запрос должен заканчиваться точкой с запятой.
	\end{itemize}
\end{frame}

\begin{frame}{Графическая утилита}
	\begin{itemize}
		\item Я выбрал \href{http://sqlitebrowser.org/}{DB Browser for SQLite}.
		\item Иногда проще смотреть на таблице в графической оболочке, чем в консоли.
		\item Может открывать файлы с БД, все изменения идут в памяти.
		\item Можно откатывать изменения кнопкой <<Revert Changes>> до последнего сохранения.
		\item Можно сохранять изменения в файл кнопкой <<Write Changes>>.
		\item Показывает таблицы, их структура, позволяет выполнять произвольные запросы.
	\end{itemize}
\end{frame}

\begin{frame}[fragile]{Python}
\begin{minted}{python}
with sqlite3.Connection("literacy.sqlite3") as db:
  cursor = db.execute("SELECT * FROM Country LIMIT 3")
  print(cursor.description)
  print(list(cursor))
  print(list(cursor))  # Что-нибудь выведет?
\end{minted}
	\begin{itemize}
		\item Терминология очень похожа во всех языках и СУБД.
		\item Обычно в языке есть стандартный интерфейс общения с любыми СУБД.
			А \textit{драйвер} СУБД реализует этот интерфейс в языке.
		\item Сначала мы устанавливаем \textit{соединение} с СУБД.
		\item Результатом запроса является \textit{курсор} "--- это такой итератор по строчкам запроса.
		\item Что возвращают запросы, кроме \t{SELECT} "--- зависит от СУБД.
		\item
			Иногда считается, что не запрос возвращает курсор, а надо
			сначала создать курсор, а потом в нём выполнить запрос.
	\end{itemize}
\end{frame}
