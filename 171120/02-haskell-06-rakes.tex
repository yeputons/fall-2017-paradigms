\subsection{Грабли}

\begin{frame}
	\tableofcontents[currentsection,currentsubsection]
\end{frame}

\begin{frame}[fragile]{Классы типов}
	\begin{itemize}
		\item Иногда хочется функцию полиморфную, но с ограничением на тип.
		\item Например, \t{min} должен уметь сравнивать свои аргументы.
		\item Набор свойств, которыми должен обладать тип, называют \textit{классом типа}.
		\item Подробнее разберём на следующем занятии.
		\item То, что идёт перед \t{=>} в типе в Haskell "--- это как раз ограничения на классы:
\begin{minted}{haskell}
min :: Ord a => a -> a -> a
\end{minted}
		\item \t{a} "--- любой тип, лежащий в классе \t{Ord}.
		\item Не путать с классами из ООП!
		\item Тут <<класс>> означает <<множество>>, как в математике.
	\end{itemize}
\end{frame}

\begin{frame}{Очень строгая типизация}
	% 08-01-types.hs
	\begin{itemize}
		\item Есть разные типы для вещественных и целых чисел.
		\item Оператор \t{==} между разными типами чисел не определён.
		\item Но тестом \t{2 == 2.0} это не поймать.
		\item Тип числовой константы определяется в момент компиляции: это будет либо вещественное число (класс \t{Fractional}), либо целое (класс \t{Integral}), либо любое (класс \t{Num}).
		\item Для конвертации между \t{Fractional} и \t{Integral} можно использовать \t{fromIntegral} и \t{round}.
	\end{itemize}
\end{frame}

\begin{frame}{Каррирование}
	\begin{itemize}
		\item Функция от двух аргументов $f(x, y)$ "--- то же самое, что функция от одного аргумента $f'(x)$,
			которая возвращает функцию от одного аргумента $g_x(y)=f(x, y)$.
		\item Такой переход называется \textit{каррированием}
		\item С точки зрения Haskell, функции типов \t{a -> b -> c} и \t{a -> (b -> c)} неотличимы.
		\item Все функции в Haskell "--- от одного аргумента, остальное "--- сахар.
		\item Можно проверить по синтаксису вызова.
		\item Можно даже посмотреть, что будет, если написать \t{map (+1)}.
		\item Или обнаружить, что \t{(3+)} и \t{(+) 3} возвращают одно и то же.
		\item Поэтому считается, что стрелочка в типах ассоциативна вправо.
		\item Если ожидали \t{(a -> b) -> (a -> b)}, а получили \t{(a -> b) -> a -> b} "--- это одно и то же.
	\end{itemize}
\end{frame}

\begin{frame}{Унарный минус}
	\begin{itemize}
		\item Единственный унарный оператор в Haskell.
		\item Захардкожен костылями.
		\item Ломает красоту, потому что \t{(-3)} надо интерпретировать как унарный минус, а не как оператор \t{-} с зафиксированным операндом.
		\item А \t{((-)3)} надо интерпретировать, как оператор \t{(-)} с зафиксированным первым (левым) операндом.
		\item Поэтому зафиксировать правый аргумент у бинарного минуса никак нельзя, кроме лямбда-функций.
	\end{itemize}
\end{frame}

\begin{frame}[fragile]{Как отлаживать}
	\begin{itemize}
		\item Ваш лучший друг "--- чтение кода и выписывание типов.
		\item Haskell обычно выдаёт точный символ, в котором произошла ошибка. Там и надо смотреть.
		\item Начинать лучше с самой верхней ошибки.
		\item Можно закомментировать кусок кода, а у оставшегося явно написать нужный тип:
\begin{minted}{haskell}
map' :: (a -> b) -> [a] -> [b]
map' f a b = f a  -- Ошибка компиляции
\end{minted}
		\item Проверяйте код при помощи \href{https://hackage.haskell.org/package/hlint}{hlint}.
		\item Он заодно проверяет соответствие некоторому стилю.
		\item Не все рекомендации hlint жизненно необходимо выполнять, так как единого стиля нет.
	\end{itemize}
\end{frame}
