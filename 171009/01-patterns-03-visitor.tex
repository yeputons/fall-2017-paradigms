\section{Паттерн Visitor}

\begin{frame}
	\tableofcontents[currentsection]
\end{frame}

\begin{frame}{Постановка задачи}
	Пусть есть разношёрстные классы, например, элементы синтаксического дерева языка ЯТЬ:
	\t{Scope}, \t{Function}, \t{Read}, \t{Write}.
	И ещё есть операции над этими объектами: вывести красиво отформатированный код,
	соптимизировать кусочек дерева (например, заменить \t{2+2} на \t{4}),
	скомпилировать программу.

	Вопрос: где описывать эти операции?
	\pause
	Несколько вариантов:
	\begin{enumerate}
		\item
			Добавить методы \t{pretty\_print}, \t{optimize}, \t{compile} всем классам.
		\item
			Сделать одну функцию на операцию, которая может вызывать сама себя рекурсивно
			и руками проверяет тип объекта.
		\item
			Паттерн \textit{Visitor}: сделать класс, соответствующий операции, с методами
			<<обработай \t{Scope}>>, <<обработай \t{Function}>>, и так далее.
	\end{enumerate}
\end{frame}

\begin{frame}[fragile]{Пример внутри классов}
\begin{minted}{python}
import random
class Cat:
    def pat_head(self): print("Purr!")
    def rub_belly(self): print("Don't you dare!")
    def happiness(self): return random.randint(1, 5)
    def pet(self): self.pat_head()
    def is_safe(self): return self.happiness() >= 3
class Dog:
    def pat_head(self): print("I'm happy!")
    def rub_belly(self): print("I'm very happy!")
    def tail_wagging(self): return True
    def pet(self): self.rub_belly()
    def is_safe(self): return self.tail_wagging()
\end{minted}
\end{frame}

\begin{frame}[fragile]{Пример в функциях}
\begin{minted}{python}
import random
class Cat:
    def pat_head(self): print("Purr!")
    def rub_belly(self): print("Don't you dare!")
    def happiness(self): return random.randint(1, 5)
class Dog:
    def pat_head(self): print("I'm happy!")
    def rub_belly(self): print("I'm very happy!")
    def tail_wagging(self): return True
def pet(a):
    if isinstance(a, Cat): a.pat_head()
    elif isinstance(a, Dog): a.rub_belly()
def is_safe(a):
    if isinstance(a, Cat): return a.happiness() >= 3
    elif isinstance(a, Dog): return a.tail_wagging()
\end{minted}
\end{frame}

\begin{frame}{Возможные проблемы}
	Если добавляем методы в классы, то:
	\pause
	\begin{itemize}
		\item Логика операции оказывается разнесённой по разным кускам кода.
		\item Добавили операцию "--- надо проверить, что ни в одном из классов не забыли
			(Python-то не проверяет соответствие интерфейсам).
		\item
			При добавлении операций изменяются интерфейсы классов, придётся всё перекомпилировать.
	\end{itemize}

	\pause
	Если делаем одну функцию, то:
	\pause
	\begin{itemize}
		\item Куча неприятного кода для определения типа, захочется разделить определение типа и содержательную обработку.
		\item В некоторых языках проверить тип переданного объекта в run time очень сложно (например, С "--- надо что-то руками делать).
		\item В статически типизированных языках надо ещё и изменять тип переменной.
	\end{itemize}
\end{frame}

\begin{frame}{Волшебный единорог}
	Паттерн Visitor (\textit{посетитель}):
	\begin{enumerate}
		\item
			Создаём интерфейс \t{Visitor} с функциями \t{visit\_scope}, \t{visit\_function}, \t{visit\_read}...
			\footnote{в некоторых языках (C++, Java) все методы назовут visit, а отличия будут лишь в типе аргумента}.
		\item
			Требуем, чтобы каждый класс имел функцию \t{accept(visitor)}, которая бы вызывала у параметра нужный метод.
		\item
			Для определения операции достаточно создать новый класс, реализующий интерфейс \t{Visitor}.
		\item
			Для вызова операции на элементе достаточно вызывать \t{item.accept(visitor)}.
	\end{enumerate}
\end{frame}

\begin{frame}[fragile]{Пример-1}
\begin{minted}{python}
import random
class Cat:
    def pat_head(self): print("Purr!")
    def rub_belly(self): print("Don't you dare!")
    def happiness(self): return random.randint(1, 5)
    def visit(self, v): return v.visit_cat(self)
class Dog:
    def pat_head(self): print("I'm happy!")
    def rub_belly(self): print("I'm very happy!")
    def tail_wagging(self): return True
    def visit(self, v): return v.visit_dog(self)
\end{minted}
\end{frame}

\begin{frame}[fragile]{Пример-2}
\begin{minted}{python}
class PetVisitor:
    def visit_cat(self, a): a.pat_head()
    def visit_dog(self, a): a.rub_belly()
class IsSafeVisitor:
    def visit_cat(self, a): return a.happiness() >= 3
    def visit_dog(self, a): return a.tail_wagging()

Cat().visit(PetVisitor())  # Purr!
Dog().visit(PetVisitor())  # I'm very happy!
print(Cat().visit(IsSafeVisitor()))  # random
print(Dog().visit(IsSafeVisitor()))  # True
\end{minted}
\end{frame}

\begin{frame}[fragile]{Замечание про возвращаемые значения}
	\begin{enumerate}
		\item
			В статически типизированных языках нельзя вернуть произвольное значение из методов,
			надо указывать какой-то тип.
		\item
			При этом указывать тип <<всё что угодно>> нехорошо, так как теряется безопасность.
			Фиксировать конкретный тип для всех посетителей тоже нехорошо.
		\item
			Поэтому часто считают, что \t{visit}/\t{accept} вообще ничего не возвращают.
			А в Python легко забыть \t{return} и не заметить :)
		\item
			Если посетитель что-то должен вычислять, то он хранит результат вычисления внутри себя:
\begin{minted}{python}
class IsSafeVisitor:
    def visit_cat(self, a):
        self.result = a.happiness() >= 3
    def visit_dog(self, a):
        self.result = a.tail_wagging()
\end{minted}
	\end{enumerate}
\end{frame}
