\section{Программная инженерия}

\begin{frame}
	\tableofcontents[currentsection]
\end{frame}

\begin{frame}{Как живут программы}
	\begin{itemize}
		\item Программы нужны для решения каких-то практических задач (кассовый аппарат).
		\item Программы пишутся на основе каких-то моделей реального мира (формальная постановка задачи).
		\item Реальный мир меняется (новые акции) $\Rightarrow$ модель меняется (новые правила расчётов) $\Rightarrow$ надо менять программу.
		\item Следствие: программы постоянно меняются.
		\item Программисты уходят и приходят.
		\item Следствие: одну программу пишут и дописывают разные программисты.
		\item \textbf{Одной корректности в данный момент недостаточно!}
	\end{itemize}
\end{frame}

\begin{frame}{Для кого пишутся программы}
	\begin{itemize}
		\item Корректности можно добиться разными способами.
		\item Некоторые способы быстрее работают, некоторые требуют меньше памяти.
		\item А какие-то \textit{лучше поддерживаются в будущем} (расширяются, изменяются, отлаживаются).
		\item Текущая мода: программисты дороже компьютеров.
		\item Следствие: программы пишутся для людей, которые их потом будут менять.
		\item Следствие: важны читаемость, расширяемость, очевидность поведения, документация...
	\end{itemize}
\end{frame}

\begin{frame}{Инженерные задачи}
	Аналогия:
	\begin{itemize}
		\item Предположим, вы хотите спроектировать (а потом построить) мост через реку.
		\item Наверняка конкретно такого моста, как вам надо, ещё никто и никогда не строил.
		\item Тем не менее, что-то про мосты человечество знает: основные типы, когда их использовать, что-то про материалы...
		\item Значит, мост вы проектируете не с нуля, а на какой-то основе.
		\item Но индивидуальные особенности всё равно приходится учитывать.
	\end{itemize}
\end{frame}

\begin{frame}{Задачи в программировании}
	\begin{itemize}
		\item Предположим, вы хотите написать сайт.
		\item Разумеется, ровно такого ещё никто и никогда не писал.
		\item
			Тем не менее, про сайты много чего известно: как хранить данные,
			как обрабатывать несколько запросов одновременно, какие бывают уязвимости, как лучше хранить пароли пользователей...
		\item Значит, сайт тоже можно сделать не с нуля, а на основе каких-то существующих идей.
		\item Разумеется, кроме них в сайте будет что-то индивидуальное.
		\item \textbf{В программировании вы постоянно комбинируете готовые идеи, решения и конструкции}
	\end{itemize}
\end{frame}

\begin{frame}{Примеры стандартных идей}
	\begin{itemize}
		\item Для обработки данных: разделять ввод, обработку и вывод данных.
		\item Циклы по элементам массива, а не по индексам.
		\item Структура данных <<словарь>> (хэш-таблица).
		\item Использование библиотек вместо прямой работы с ОС $\Rightarrow$ кроссплатформенность.
	\end{itemize}
\end{frame}
