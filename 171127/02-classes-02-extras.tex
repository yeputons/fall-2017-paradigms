\subsection{Прочие плюшки}
\begin{frame}[fragile]{Свои классы типов}
	Можно создать свой собственный класс:
\begin{minted}{haskell}
class HasSize a where
  size :: a -> Int

instance HasSize [a] where
  size xs = length xs

data Tree a = Nil | Node a (Tree a) (Tree a)
instance HasSize (Tree a) where
  size Nil = 0
  size (Node _ l r) = 1 + size l + size r
\end{minted}
	\begin{itemize}
		\item
			Можно определить даже для старых типов.
			В Java/C++ такое можно сделать только внешними перегруженными функциями.
		\item Новые функции можно использовать где угодно с любыми типами.
	\end{itemize}
\end{frame}

\begin{frame}[fragile]{Стандартные классы}
	\begin{itemize}
		\item \t{Show} "--- то, что можно вывести на экран.
		\item \t{Eq} "--- операторы \t{==} и \t{/=}.
		\item \t{Ord} "--- операторы \t{<}, \t{<=} и прочие.
		\item \t{Functor} "--- структура данных, на которой есть \t{map}.
		\item \t{Foldable} "--- структура данных, на которой есть \t{foldr} (по сути, умеет разворачиваться в список).
		\item Для первых трёх Haskell умеет сам генерировать адекватные реализации, если попросить:
\begin{minted}{haskell}
data List a = Empty | Cons a (List a)
            deriving (Show, Eq, Ord)
\end{minted}
		\item Порядок <<лексикографический>> (более ранний конструктор меньше).
	\end{itemize}
\end{frame}

\begin{frame}[fragile]{Зависимости классов}
	\begin{itemize}
		\item Как определить \t{Ord}? Хотим, чтобы тип класса \t{Ord} автоматически подходил везде, где нужен лишь \t{Eq}.
		\item Можно скопировать \t{==} и \t{/=} в \t{Ord}, но тогда:
			\begin{enumerate}
				\item Можно случайно реализовать и \t{Ord}, и \t{Eq}, причём по-разному.
				\item Получаем <<утиную типизацию>>: один класс вкладывается в другой, если есть функции с такими же названиями.
				\item Это нехорошо: придётся следить, что названия функций вообще нигде не пересекаются.
			\end{enumerate}
		\item Можно сказать, что \t{Ord} определяет только новые операторы, но тогда мы можем определить тип с \t{<}, но без \t{==}, что странно.
		\item Решение: класс \t{Ord} требует, чтобы тип также был в классе \t{Eq}:
\begin{minted}{haskell}
class Eq a => Ord a where
  (<), (<=), (>=), (>) :: a -> a -> Bool
\end{minted}
		\item Если где-то пишем контекст \t{Ord a}, то \t{Eq a} появляется неявно.
		\item В частности, в реализациях по умолчанию в \t{Ord a} можно использовать \t{==} и \t{/=}.
	\end{itemize}
\end{frame}

\begin{frame}[fragile]{Автовывод контекста}
\begin{minted}{haskell}
-- Ord a => a -> a -> a
max' a b = if a > b then a else b

-- (Functor f, Eq a) => a -> f a -> f (Maybe a)
removeByValue x ys = fmap f ys
  where
    f y | x == y    = Nothing
        | otherwise = Just y
\end{minted}
	Если в файле не видно разных функций с одинаковым названием из разных классов, то компилятор может автоматически вывести ограничения на типы (контекст).
\end{frame}

\begin{frame}[fragile]{Резюме}
	\begin{itemize}
		\item Альтернатива классам типов "--- интерфейсы из ООП или перегрузки функций.
		\item Перегрузки функций не отражают связи между разными функциями (вроде \t{==} и \t{/=}).
		\item Интерфейсы из ООП \textit{обычно} надо определять в момент создания каждого типа (не добавить интерфейс к уже существующему).
		\item Интерфейсы из ООП \textit{обычно} не позволяют делать реализации по умолчанию "--- надо писать руками.
		\item Классы типов всё это позволяют.
		\item Компилятор умеет автоматически выводить нужный контекст.
		\item В Haskell классов типов используется везде, где есть хотя бы доля обобщаемости.
	\end{itemize}
\end{frame}

\begin{frame}[fragile]{Упражнение}
	\begin{enumerate}
		\item Пусть имеется следующее дерево со значениями в листьях и произвольным числом детей:
\begin{minted}{haskell}
data Tree a = Leaf a | Node [Tree a] deriving Show
\end{minted}
		\item Добавьте Tree в класс \t{Eq}. Подсказка: списки там уже есть.
		\item Пусть есть такой класс для структур, в которых можно развернуть порядок элементов:
\begin{minted}{haskell}
class Reversible f where
   reverse' :: f a -> f a
\end{minted}
		\item Добавьте списки в этот класс:
\begin{minted}{haskell}
instance Reversible [] where
\end{minted}
		\item Добавьте дерево в этот класс:
\begin{minted}{haskell}
instance Reversible Tree where
\end{minted}
	\end{enumerate}
\end{frame}
