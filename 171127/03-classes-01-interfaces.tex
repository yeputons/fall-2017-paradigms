\section{Классы типов}
\subsection{Что и зачем}

\begin{frame}
	\tableofcontents[currentsection,currentsubsection]
\end{frame}

\begin{frame}[fragile]{Pattern Matching и \t{==}}
\begin{minted}{haskell}
data IntList = Empty | Cons Int IntList deriving Show
a = Cons 1 (Cons 2 Empty)
b = Cons 1 (Cons 2 Empty)
c = Cons 1 (Cons 3 Empty)

isA :: IntList -> Bool
isA (Cons 1 (Cons 2 Empty)) = True
isA _                       = False

isA a -- True
isA b -- True
isA c -- False
a == b  -- ошибка компиляции?
a == c  -- ошибка компиляции?
\end{minted}
\end{frame}

\begin{frame}[fragile]{Eq}
	\begin{itemize}
		\item Pattern Matching "--- конструкция на уровне языка.
		\item \t{==} "--- просто некоторая функция с таким названием.
		\item В C++ мы бы написали перегрузку функции/оператора.
		\item В Haskell пишем так:
\begin{minted}{haskell}
instance Eq IntList where
  Empty       == Empty       = True
  (Cons x xs) == (Cons y ys) = (x == y) && (xs == ys)
  _           == _           = False

a == b  -- True
a == c  -- False
b == c  -- False
Empty == Empty           -- True
Empty /= (Cons 1 Empty)  -- True, /= тоже работает
\end{minted}
	\end{itemize}
\end{frame}

\begin{frame}[fragile]{class Eq}
	\begin{itemize}
		\item \t{Eq} "--- это \textit{класс типов}, который описывает, что к типам можно применять определённые функции:
\begin{minted}{haskell}
class Eq a where
  (==) :: a -> a -> Bool
  (/=) :: a -> a -> Bool
\end{minted}
		\item Говорим, что тип \t{a} лежит в классе \t{Eq} тогда и только тогда, когда для него есть функции \t{(==)} и \t{(/=)}
		\item Класс типов "--- это такой <<интерфейс>> для типов.
		\item Некоторые функции требуют, чтобы параметры были в определённых классах:
\begin{minted}{haskell}
lookup :: Eq a => a -> [(a, b)] -> Maybe b
\end{minted}
		\item Слово \t{instance} на предыдущем слайде добавляло \t{IntList} в класс \t{Eq}.
		\item Не путать с классами объектов из ООП!
	\end{itemize}
\end{frame}

\subsection{Для параметризованных типов}
\begin{frame}[fragile]{Класс Eq для списков}
	\begin{itemize}
		\item Пусть есть свой класс для списков:
\begin{minted}{haskell}
data List a = Empty | Cons a (List a)
\end{minted}
		\item Разумно считать, что списки равны, если равны элементы:
\begin{minted}{haskell}
instance Eq (List a) where
  Empty       == Empty       = True
  (Cons x xs) == (Cons y ys) = (x == y) && (xs == ys)
  _ == _                     = False
\end{minted}
		\item Не скомпилируется, потому что элементы произвольного типа \t{a} нельзя сранивать.
		\item Надо добавить \textit{контекст} "--- сказать, что списки можно сравнивать только если можно сранивать элементы:
\begin{minted}{haskell}
instance Eq a => Eq (List a) where
\end{minted}
	\end{itemize}
\end{frame}
