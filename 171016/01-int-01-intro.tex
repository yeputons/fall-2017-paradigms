\section{Целые числа}
\subsection{Физическая часть}

\begin{frame}
	\tableofcontents[currentsection,currentsubsection]
\end{frame}

\begin{frame}
	Упрощённое представление о происходящем в <<железе>>:
	\begin{enumerate}
		\item Любой сигнал (в том числе бит) "--- это напряжение на проводе.
		\item Два уровня напряжения распознавать проще, чем три.
		\item Но три \href{https://ru.wikipedia.org/wiki/\%D0\%A1\%D0\%B5\%D1\%82\%D1\%83\%D0\%BD\%D1\%8C\_(\%D0\%BA\%D0\%BE\%D0\%BC\%D0\%BF\%D1\%8C\%D1\%8E\%D1\%82\%D0\%B5\%D1\%80)}{тоже было}, не прижилось.
		\item Вся логика построена на основе бинарных функций <<И>>, <<ИЛИ>> и остальных (\textit{гейты})
		\item Чем меньше гейтов "--- тем быстрее работает, тем меньше схема.
		\item Числа надо складывать, вычитать, умножать, делить, сравнивать на равенство и меньше/больше.
	\end{enumerate}
\end{frame}

\subsection{Типы данных}
\begin{frame}{Играем в игру}
	Чему соответствует бинарная запись в таблице ниже?

	Используйте калькулятор или Python: \t{0b0100} и \t{int('1111', 2)}.
	\begin{center}
		\pause
		\begin{tabular}{|c|c|}
			\hline
			\t{0001 0110} & \pause 22 \\\hline\noalign{\pause}
			\t{1000 0010} & \pause 130 \\\hline\noalign{\pause}
			\t{1000 0010} & \pause -126 \\\hline\noalign{\pause}
			\t{0011 0000} & \pause 48 \\\hline\noalign{\pause}
			\t{0011 0000} & \pause '0' \\\hline\noalign{\pause}
			\t{1100 0011} & \pause \t{0xC3} \\\hline\noalign{\pause}
			\t{1100 0011} & \pause \t{ret} \\\hline\noalign{\pause}
			\t{0110 1000} & \pause 22 \\
			\hline
		\end{tabular}
		\pause
	\end{center}
	Мораль: битовое представление ничего не говорит, если мы не договорились о том,
	как его интерпретировать (<<тип>>).

	Более того, представлений у одной и той же сущности может быть в
	некотором смысле много (\t{0xC3}, 195, \t{ret}).
\end{frame}

\begin{frame}{Ликбез-1}
	\begin{itemize}
		\item Основные типы чисел: целое, с фиксированной запятой, с плавающей запятой.
		\item Про строки и кодировки не говорим, там тоже довольно весело и интересно.
		\item Железо сейчас в основном поддерживает целые числа и с плавающей запятой.
		\item Железо умеет получать доступ к байту в памяти по его \textit{адресу}.
		\item Считаем, что адрес "--- это некоторое целое неотрицательное число.
	\end{itemize}
\end{frame}

\begin{frame}{Ликбез-2}
	\begin{itemize}
		\item
			Железо не может адресовать что-то внутри байта (биты).
		\item
			Но мы можем выполнять какие-то арифметические операции с байтами.
		\item
			Про порядок бит внутри байта говорить бессмысленно "--- мы никак его не проверим, у нас есть только арифметические операции.
		\item
			Будем рисовать \textit{младшие}/\textit{менее значимые} биты справа, как будто нормальные числа):
			\[ \t{0001 0010}_2 = 18_{10} \]
		\item
			Если какая-то конструкция занимает несколько байт подряд, то важно, в каком порядке они идут.
		\item
			Будем рисовать память слева направа: нулевой байт, первый...
	\end{itemize}
\end{frame}
