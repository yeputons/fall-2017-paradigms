\subsection{Синтаксис сравнений}
\begin{frame}
	\tableofcontents[currentsection,currentsubsection]
\end{frame}

\begin{frame}[fragile]{Сравнения-1}
\begin{minted}{python}
class Natural:
    def __init__(self, value):
        self.value = value
    def __lt__(self, other):  # Less than
        return self.value < other.value
    def __le__(self, other):  # Less or equal
        return self.value <= other.value
    def __eq__(self, other):
        return self.value == other.value
ONE, TWO = map(Natural, [1,2])
print(ONE < TWO, ONE > TWO)    # True False
print(ONE <= TWO, ONE >= TWO)  # True False
print(ONE == TWO, ONE != TWO)  # False True
print(ONE == ONE, ONE != ONE)  # True False
\end{minted}
\end{frame}

\begin{frame}[fragile]{Сравнения-2}
	\begin{itemize}
		\item
			Операторы \t{\_\_lt\_\_} и \t{\_\_gt\_\_} считаются отражениями друг друга.
			Почти как \t{\_\_add\_\_} и \t{\_\_radd\_\_}.
		\item
			Оператор \t{\_\_ne\_\_} по умолчанию берёт отрицание от \t{\_\_eq\_\_}.
		\item
			Оператор \t{\_\_lt\_\_} автоматически из \t{\_\_le\_\_} и \t{\_\_eq\_\_} не выводится.
	\end{itemize}
	Можно использовать \textit{декоратор} \href{https://docs.python.org/3/library/functools.html\#functools.total\_ordering}{\t{total\_ordering}}:
\begin{minted}{python}
from functools import total_ordering
@total_ordering  # Магия.
class Natural:
    # Надо задать метод __eq__ и один из четырёх сравнивающих.
    #
    # Остальное сгенерируется.
\end{minted}
\end{frame}

\begin{frame}[fragile]{Хэш-таблицы-1}
\begin{minted}{python}
a = { Natural(1): 1 }  # TypeError: unhashable type

class Natural:
    def __init__(self, value):
        self.value = value
    def __eq__(self, other):
        return self.value == other.value
    def __hash__(self):
        return hash(self.value)
    def __repr__(self):
        return "Natural({})".format(self.value)

a = {Natural(x): x for x in range(5)}
print(a)
\end{minted}
\end{frame}

\begin{frame}{Хэш-таблицы-2}
	\begin{itemize}
		\item
			\t{\_\_hash\_\_} вызывается функцией \t{hash}, когда элемент кладут в хэш-таблицу.
			Должна вернуть \t{int}.
		\item
			Требование: если \t{a == b}, то \t{hash(a) == hash(b)} (в обратную сторону необязательно).
		\item
			Если определён \t{\_\_hash\_\_}, то обязательно определить \t{\_\_eq\_\_}.
		\item
			Для помещения в хэш-таблицу методы \t{\_\_lt\_\_} необязательны.
		\item
			В языке Java идеология похожа: методы \t{equals()} и \t{hashCode()}.
		\item
			Если объект может измениться (\textit{мутабельный}), то \t{\_\_hash\_\_} определять не стоит.
			Почему?
			\pause
			Потому что если он изменится, пока лежит в хэш-таблице, она об этом не узнает и сломается.
	\end{itemize}
\end{frame}

\begin{frame}[fragile]{Хэш-таблицы-3}
	Чем плоха реализация строки ниже?
\begin{minted}{python}
class Str:
    def __init__(self, value): self.value = value
    def __eq__(self, other): return self.value == other.value
    def __hash__(self):
        result = 0
        for c in self.value:
            result = result * 239017 + ord(c)
        return result
\end{minted}
	\pause
\begin{minted}{python}
print(hash("a" * 1000))
print(hash(Str("a" * 1000)))
print(hash("a" * 100000))
print(hash(Str("a" * 100000)))
\end{minted}
	\pause
	Тормозит, потому что в Python \t{int} автоматически преобразуется в длинную арифметику,
	а не переполняется.
\end{frame}

\begin{frame}[fragile]{Упражнение}
	Реализуйте класс \t{Natural} для хранения натуральных чисел с поддержкой сложения, умножения (в том числе с int), хэширования, вывода на экран и сравнений:
\begin{minted}{python}
class Natural:
    # ... ваш код здесь ...

# Тесты:
l = list(map(Natural, [5, 2, 3, 4, 1, 4]))
print(Natural(2) + Natural(3))
print(20 * Natural(100) + 5)
print(sum(l))
print(sorted(l))
d = {Natural(1): 10, Natural(2): 20}
print(list(d.keys()))  # [Natural(1), Natural(2)]
\end{minted}
\end{frame}
