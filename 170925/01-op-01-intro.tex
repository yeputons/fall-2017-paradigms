\section{Перегрузка операторов}
\subsection{Мотивация}

\begin{frame}
	\tableofcontents[currentsection]
\end{frame}

\begin{frame}[fragile]{Зачем перегружать?}
	\begin{itemize}
		\item Задание требуемой семантики у своих объектов:	
\begin{minted}{python}
class Foo:
    def __init__(self, value):
        self.value = value
Foo("hello") == Foo("hello")  # False
\end{minted}

		\item Упрощение кода с математическими объектами:
\begin{minted}{python}
# До
res = a.multiply(x).add(b.multiply(y)) \
    .add(c).multiply(5).add(2)
middle = vector1.add(vector2).multiply(0.5)
	
# После
res = (a * x + b * y + c) * 5 + 2
middle = (vector1 + vector2) / 2
\end{minted}
	\end{itemize}
\end{frame}

\begin{frame}[fragile]{Почему не перегружать?}
	\pause
	\begin{itemize}
		\item
			Разное поведение у похожих типов.
			Например: определим \t{/} как целочисленное деление.
			Лучше определить только \t{//}.
			\pause
		\item
\begin{minted}{cpp}
line a = /* ... */, b = /* ... */;
if (a || b) { /* ... */ }
\end{minted}
			\pause
			Можно сказать, что \t{||} возвращает, параллельны ли прямые.
			Полностью изменяется семантика оператора.
			\pause
		\item
			Скрывает сложные операции при чтении кода.
			Может мешать при отладке и поиске медленных мест "--- нет явного вызова функции:
\begin{minted}{python}
a = 10 ** 10000
b = 10 ** 10000
# ...
result = a * b  # Почему же тормозит?
\end{minted}
		\item
			Неочевидное поведение: \t{vec1 * vec2}.
			\pause
			Векторное или скалярное произведение?
	\end{itemize}
\end{frame}

\subsection{Немного магии}
\begin{frame}{Магические методы}
	\begin{itemize}
		\item
			\textit{Магическим} зовётся метод, название которого начинается и заканчивается на \t{\_\_}.
			Например, \t{\_\_init\_\_} или \t{\_\_str\_\_}.
		\item Ничего магического, кроме предназначения, в них нет.
		\item Напрямую их вызывать не стоит!
		\item Перечислены в документации по группам. Объекты могут прикидываться:
			\begin{enumerate}
				\item \href{https://docs.python.org/3/reference/datamodel.html\#emulating-numeric-types}{Числами}.
				\item \href{https://docs.python.org/3/reference/datamodel.html\#object.\_\_lt\_\_}{Чем-то, что можно сравнивать}.
				\item \href{https://docs.python.org/3/reference/datamodel.html\#emulating-callable-objects}{Функциями}.
				\item \href{https://docs.python.org/3/reference/datamodel.html\#emulating-container-types}{Коллекциями} (массив, словарь, множество...).
				\item \href{https://docs.python.org/3/library/stdtypes.html\#typeiter}{Итераторами} (обслуживают цикл \t{for}).
				\item \href{https://docs.python.org/3/reference/datamodel.html\#with-statement-context-managers}{Чем-что, что можно автоматически закрывать} (файл, сетевое соединение).
				\item И ещё много чем.
			\end{enumerate}
		\item Какая-то информация гуглится и \href{https://pythonworld.ru/osnovy/peregruzka-operatorov.html}{на русском}.
	\end{itemize}
\end{frame}
