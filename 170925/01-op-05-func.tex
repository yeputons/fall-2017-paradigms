\subsection{Притворяемся функцией}
\begin{frame}
	\tableofcontents[currentsection,currentsubsection]
\end{frame}

\begin{frame}[fragile]{Притворяемся функцией}
\begin{minted}{python}
class Summer:  # Сумма, а не лето :(
    def __init__(self, k):
        self.k = k
    def __call__(self, *args):
        return self.k * sum(args)
s = Summer(3)
print(s())       # 0
print(s(1))      # 3
print(s(1, 10))  # 33
\end{minted}
	Всё, что имеет метод \t{\_\_call\_\_}, может быть вызвано.
	И наоборот:
\begin{minted}{python}
def foo(): print("foo")
print(foo.__call__)
foo.__call__()
print(foo.__call__.__call__)
\end{minted}
\end{frame}

\begin{frame}{Но зачем?}
	\begin{itemize}
		\item
			Объект каком-то смысле представляет собой функцию.
			Например: <<преобразование плоскости>>, <<логгер>> или <<выражение от одной переменной>>.
		\item
			В некоторых других языках это единственный способ сделать функцию с некоторым внутренним состоянием (кроме глобальных переменных).
		\item
			Все проблемы с перегрузкой операторов остаются.
			Не злоупотребляйте!
	\end{itemize}
\end{frame}
