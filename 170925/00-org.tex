\begin{frame}[t]{Организационное}
	\begin{itemize}
	\item Я студент четвёртого курса бакалавриата СПб АУ (программист).
	\item Имена: <<извините, пожалуйста>>, <<Егор>>, <<Егор Фёдорович>>.
	\item Можно на <<вы>>, можно на <<ты>>.
	\item
		\begin{tabular}{rl}
			E-mail: & \href{mailto:\EgorEmail?subject=[paradigms]}{\t{\EgorEmail}} \\
			Тема e-mail: & \t{[paradigms]...} \\
			ВК: & \href{https://vk.com/\EgorVk}{\t{\EgorVk}} \\
			Telegram: & \href{https://telegram.me/\EgorTelegram}{\t{\EgorTelegram}}
		\end{tabular}
	\item
		Домашние задания общие с остальными подгруппами.
	\item
		Решения надо присылать мне.
		Если уже прислали кому-то ещё "--- перенаправлять не надо.
	\item
		Вопросы по текущей теме или смежным можно задавать в ходе рассказа на занятии.
	\item
		Вопросы по остальным темам и предметам лучше в оффлайне.
	\item
		Все материалы "--- на SEWiki (пополняется).
	\item
		Как лучше оповещать об обновлениях?
	\end{itemize}
\end{frame}

\begin{frame}[t]{Зачёт и проверка}
	\begin{itemize}
		\item Надо набрать хотя бы половину баллов от максимума.
		\item Надо набрать строго положительные баллы в каждой домашке.
		\item Обычно: одна домашка "--- одна тема.
		\item В некоторых домашках будет несколько подзаданий; могут быть сложные.
		\item Оценивается в первую очередь корректность и \textit{точное} соответствие заданию.
		\item Если решение проходит автоматически проверки, вы получаете половину баллов.
		\item Оставшуюся половину получаете за субъективные параметры (<<стиль>>).
		\item Досдавать можно и нужно.
	\end{itemize}
\end{frame}

\begin{frame}[t]{Обратная связь}
	\begin{itemize}
		\item
			Готов обсуждать и даже менять по согласованию критерии оценки, правила игры, оргмоменты.
		\item
			Любая критика и жалобы на жизнь также приветствуются.
			Особенно если есть предложения <<как лучше>>.
		\item
			Можно писать и передавать коллективные письма.
		\item
			О планируемых завалах (неделя коллоквиумов/презентация проектов/отдых) лучше предупреждать заранее.
		\item
			Кому (не)комфортно читать технический английский?
		\item
			Чего вы ждёте от этого курса? От университета?
		\item
			Делитесь тайными знаниями не только с товарищами, но и со мной.
			Тогда я знаю, что я упустил на паре.
	\end{itemize}
\end{frame}
